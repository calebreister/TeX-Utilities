%Characters
\usepackage{fontspec}
\usepackage{amsmath, amsfonts, amssymb, amstext}
\usepackage{commath}
\usepackage{xfrac} %\sfrac (slanted fraction)
\usepackage{gensymb}
\usepackage{cancel} %\cancel & \cancelto commands
\usepackage{steinmetz} %Phasor angle notation

%Utilities
\usepackage{array}
\usepackage{graphicx}
\usepackage{float}
\usepackage[margin=1in]{geometry}
\usepackage[parfill]{parskip}
\usepackage{hyperref}
\usepackage{comment}

%Programming
\usepackage{xparse}
\usepackage{ifthen}
\usepackage{luacode}

%Import Lua modules
\directlua{require('util')}
\directlua{require('csv')}

%Redfines commands in some AMS packages
%An extensive list of characters: texdoc unimath-symbols
\usepackage{unicode-math}

\def\arraystretch{1.5} %Give tabular and array environments internal padding

%Enable hyperlinked TOC
\hypersetup{
	%linktocpage,
    colorlinks,
    citecolor=black,
    filecolor=black,
    linkcolor=black,
}

%Graphics configuration
\DeclareGraphicsExtensions{.pdf,.png,.jpg}
\graphicspath{{./img/}}

%Character overrides
\defaultfontfeatures{Ligatures=TeX}
\renewcommand{\j}{\ensuremath{\mitBbbj}} % doublestruck imaginary ⅉ symbol
\newcommand{\e}{\mitBbbe} %doublestruck exponential e symbol

%Macros
\newcommand{\note}[1]{\textbf{Note: }#1}
\newcommand{\quickimg}[1]{\includegraphics[width=\textwidth]{#1}\smallskip}
\newcommand{\E}[1]{\ensuremath{{\scriptstyle\mathsf{E}{#1}}}} %Scientific notation
\newcommand{\pr}[1]{\ensuremath{\left( #1 \right)}} %Parenthisized expression

%"Unit" macro for defining SI units. Math mode characters are acceptable.
%Note:this macro injects a space between the unit and whatever is before it in math mode, but not in text.
\newcommand{\n}[1]{
	\ifmmode \mathrm{\ #1}
	\else $\mathrm{#1}$
	\fi
}

\newcommand{\eqntable}[1]{
	%This is a tabular block that is useful for creating a list of
	%equations. The name of each equation goes on the right and
	%the equation itself goes on the left.
	\begin{tabular}{ r | >{$\displaystyle}l<{$} }
		#1
	\end{tabular}
}

\DeclareDocumentCommand{\intg}{O{} O{} m m}{
	%Integral with optional limits.
	%Enables cleaner integration expressisions
	%#1 & #2: optional limits of integration
	%#3: integrand
	%#4: variable of integration
	\ensuremath{
		\int_{#1}^{#2}#3\,d{#4}
	}
}

\DeclareDocumentCommand{\limits}{m m m}{
	%Vertical bar with upper and lower limits
	%#1: expression
	%#2: lower limit
	%#3: upper limit
	\left. #1 \right|_{#2}^{#3}
}

\DeclareDocumentCommand{\sidebyside}{O{.5} O{.5} m m O{} O{}}{
	%This is an xparse macro that places two images side by side.
	%#1 & #2: optionally define the fraction of \textwidth each  image should use.
	%#3 & #4: define the image files to use.
	%#5: optional caption
	%#6: optional label
	\begin{figure}[H]
		\begin{minipage}{#1\textwidth}
			\includegraphics[width=\textwidth]{#3}
		\end{minipage}
		\begin{minipage}{#2\textwidth}
			\includegraphics[width=\textwidth]{#4}
		\end{minipage}
		%Show caption if the user provides one.
		\ifthenelse{\equal{#5}{}}{}{\caption{#5}}
		\ifthenelse{\equal{#6}{}}{}{\label{#6}}
	\end{figure}
}



%%%%%%%%%%%%%%%%%%%%%%%%%%%%%%%%%%%%%%%%%%%%%%%%%%%%%%%%%%%%%%%%%%%%%%%%%%%%%%%%
%Preamble of the US Constitution
%We the People of the United States, in Order to form a more perfect Union, establish Justice, insure domestic Tranquility, provide for the common defence, promote the general Welfare, and secure the Blessings of Liberty to ourselves and our Posterity, do ordain and establish this Constitution for the United States of America.
