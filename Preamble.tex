%Characters
\usepackage{fontspec}
\usepackage{amsmath, amsfonts, amssymb, amstext}
\usepackage{commath}
\usepackage{xfrac} %\sfrac (slanted fraction)
\usepackage{gensymb}
\usepackage{cancel} %\cancel & \cancelto commands
\usepackage{steinmetz} %Phasor angle notation

%Utilities
\usepackage{array}
\usepackage{graphicx}
\usepackage{float}
\usepackage[margin=1in]{geometry}
\usepackage[parfill]{parskip}
\usepackage{hyperref}
\usepackage{comment}

%Programming
\usepackage{xparse}
\usepackage{ifthen}
\usepackage{luacode}
\directlua{require('csv')}

%Redfines commands in some AMS packages
%An extensive list of characters: texdoc unimath-symbols
\usepackage{unicode-math}

\def\arraystretch{1.5} %Give tabular and array environments internal padding

%Enable hyperlinked TOC
\hypersetup{
	%linktocpage,
    colorlinks,
    citecolor=black,
    filecolor=black,
    linkcolor=black,
}

%Graphics configuration
\DeclareGraphicsExtensions{.pdf,.png,.jpg}
\graphicspath{{./img/}}

%Character overrides
\defaultfontfeatures{Ligatures=TeX}
\renewcommand{\j}{\ensuremath{\mitBbbj}} % doublestruck imaginary ⅉ symbol
\newcommand{\e}{\mitBbbe} %doublestruck exponential e symbol
\newcommand{\E}[1]{\ensuremath{{\scriptstyle\mathsf{E}{#1}}}} %Scientific notation
\newcommand{\pr}[1]{\ensuremath{\left( #1 \right)}} %Parenthisized

%Macros
\newcommand{\note}[1]{\textbf{Note: }#1}
\newcommand{\quickimg}[1]{\includegraphics[width=\textwidth]{#1}\smallskip}

\newcommand{\eqntable}[1]{
	%This is a tabular block that is useful for creating a list of
	%equations. The name of each equation goes on the right and
	%the equation itself goes on the left.
	\begin{tabular}{ r | >{$\displaystyle}l<{$} }
		#1
	\end{tabular}
}

\DeclareDocumentCommand{\intg}{O{} O{} m m}{
	%Integral with optional limits.
	%Enables cleaner integration expressisions
	%#1 & #2: optional limits of integration
	%#3: integrand
	%#4: variable of integration
	\ensuremath{
		\int_{#1}^{#2}#3\,d{#4}
	}
}

\DeclareDocumentCommand{\limits}{m m m}{
	%Vertical bar with upper and lower limits
	%#1: expression
	%#2: lower limit
	%#3: upper limit
	\left. #1 \right|_{#2}^{#3}
}

\DeclareDocumentCommand{\sidebyside}{O{.5} O{.5} m m O{} O{}}{
	%This is an xparse macro that places two images side by side.
	%#1 & #2: optionally define the fraction of \textwidth each  image should use.
	%#3 & #4: define the image files to use.
	%#5: optional caption
	%#6: optional label
	\begin{figure}[H]
		\begin{minipage}{#1\textwidth}
			\includegraphics[width=\textwidth]{#3}
		\end{minipage}
		\begin{minipage}{#2\textwidth}
			\includegraphics[width=\textwidth]{#4}
		\end{minipage}
		%Show caption if the user provides one.
		\ifthenelse{\equal{#5}{}}{}{\caption{#5}}
		\ifthenelse{\equal{#6}{}}{}{\label{#6}}
	\end{figure}
}
